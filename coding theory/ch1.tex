\chapter{Basics}

%%%%%%%%%%%%%%%%%
% Section 1%%%%%%%%%%%
%%%%%%%%%%%%%%%%%


\section{Basic Notations and Definitions}
  \begin{definition}{Code}{def:code}
  	A \textit{code} $\cC$ is a subset of $ \Sigma^n$ where $\Sigma$ is an alphabet, where $n$ is the block length of $\cC$. We typically use $q$ to denote $|\Sigma|$.
  \end{definition}
  Another way to view the definition of a code to be a map $\cC: [M]\to \Sigma^n$, where $M = |\mathcal{C}|$. 
  \begin{definition}{Dimension of a code}{def:dim}
  	\textit{Dimension} of a code $\cC$, denoted as $k$, is defined as the following way, 
  	\[
  	k := \log_q|\cC|.
  	\]
  \end{definition}
  	\begin{remark} Note that, 
\begin{enumerate}
		\item For any $\cC\subseteq\Sigma^n$, $k\leq n$.
	\item $k$ can be non-integral.
\end{enumerate}
\end{remark}

One way to quantify \textit{Redundancy} in a code is via its rate.
\begin{definition}{Rate of a Code}{def:rate}
	\textit{Rate} of a code $\cC$ of block length $n$ and dimension $k$, denoted as $R$, is defined as 
	\[
	R := \frac{k}{n}.
	\]
\end{definition}

\begin{example}
Define a code $\cC\subseteq\{0,1\}^5$ that maps a binary string $(x_1, x_2, x_3, x_4)$ to $(x_1, \ldots, x_4, x_1\oplus\ldots\oplus x_4)$.
\end{example}


%%%%%%%%%%%%%%%%%%
% Section 2%%%%%%%%%%%%
%%%%%%%%%%%%%%%%%%

\section{Formalizing Error Correction}
\begin{definition}{Encoding \& Decoding Functions}{def:encode-decode}
	\begin{itemize}
\item 	Let $\cC\subseteq \Sigma^n$. An equivalent description of the code $\cC$ is an injective mapping $E : [|\cC|]\to \Sigma^n$ called the \textit{encoding function}.
\item Let $\cC\subseteq \Sigma^n$ be a code. A mapping $D:\Sigma^n\to[|\cC|]$ is called a \textit{decoding function}.
	\end{itemize}
\end{definition}

We can refer to the following figure from now on.

\subsection*{Quantifying Error}
\begin{definition}{Hamming Distance}{def:ham-dist}
For any $a, b\in\Sigma^n$, \textit{hamming distance} of $a,b$ is defined as
\[
\Delta(a,b) := \# \{i\in[n] : a_i\neq b_i\}
\]
where $a = (a_1,\ldots, a_n)$ and $b = (b_1, \ldots, b_n)$. We also define \textit{relative hamming distance} as 
\[
\delta (a,b) := \frac{1}{n}\Delta(a,b).
\]
\end{definition}
\begin{remark}
We can verify easily that this $\Delta$ is actually a distance on $\Sigma^n$.
\end{remark}

\begin{definition}{Hamming Weight}{def:hamm-weight}
We define \textit{hamming weight} of an element $v\in\Sigma^n$ as, $wt(c) := \Delta(0,v)$.
\end{definition}

Refering to the above figure, we define the error for that transmitted codeword to be $\Delta(c,y)$.

\section{Performance of an Error Correcting Code}

\begin{definition}{t-Error Channel \& t-Error Correcting Code}{def:t-error,erasure}
	\begin{itemize}
\item \text{(t-Error Channel) : } An $n$ symbol \textit{t-Error Channel} is a relation $ch : \Sigma^n\to \Sigma$ such that $\forall c\in\Sigma^n$, 
\[
\Delta(c, ch(c))\leq t.
\]
\item \text{(t-Error Correcting Code): } Let $\cC\in\Sigma^n$ is a \textit{t-Error Correcting code} if $\forall$ t-Error Channel $ch$ and $\forall m\in[|\cC|]$ ,
\[
D(ch(E(m))) = m.
\]
	\end{itemize}

\end{definition}
\begin{example}
$\cC_{3,rep}$ is a 1-error correcting code.
\end{example}
