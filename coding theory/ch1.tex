\chapter{Basics}

%%%%%%%%%%%%%%%%%
% Section 1%%%%%%%%%%%
%%%%%%%%%%%%%%%%%


\section{Basic Notations and Definitions}
  \begin{definition}{Code}{def:code}
  	A \textit{code} $\cC$ is a subset of $ \Sigma^n$ where $\Sigma$ is an alphabet, where $n$ is the block length of $\cC$. We typically use $q$ to denote $|\Sigma|$.
  \end{definition}
  Another way to view the definition of a code to be a map $\cC: [M]\to \Sigma^n$, where $M = |\mathcal{C}|$. 
  \begin{definition}{Dimension of a code}{def:dim}
  	\textit{Dimension} of a code $\cC$, denoted as $k$, is defined as the following way, 
  	\[
  	k := \log_q|\cC|.
  	\]
  \end{definition}
  	\begin{remark} Note that, 
\begin{enumerate}
		\item For any $\cC\subseteq\Sigma^n$, $k\leq n$.
	\item $k$ can be non-integral.
\end{enumerate}
\end{remark}

One way to quantify \textit{Redundancy} in a code is via its rate.
\begin{definition}{Rate of a Code}{def:rate}
	\textit{Rate} of a code $\cC$ of block length $n$ and dimension $k$, denoted as $R$, is defined as 
	\[
	R := \frac{k}{n}.
	\]
\end{definition}

\begin{example}
Define a code $\cC\subseteq\{0,1\}^5$ that maps a binary string $(x_1, x_2, x_3, x_4)$ to $(x_1, \ldots, x_4, x_1\oplus\ldots\oplus x_4)$.
\end{example}


%%%%%%%%%%%%%%%%%%
% Section 2%%%%%%%%%%%%
%%%%%%%%%%%%%%%%%%

\section{Formalizing Error Correction}
\begin{definition}{Encoding \& Decoding Functions}{def:encode-decode}
	\begin{itemize}
\item 	Let $\cC\subseteq \Sigma^n$. An equivalent description of the code $\cC$ is an injective mapping $E : [|\cC|]\to \Sigma^n$ called the \textit{encoding function}.
\item Let $\cC\subseteq \Sigma^n$ be a code. A mapping $D:\Sigma^n\to[|\cC|]$ is called a \textit{decoding function}.
	\end{itemize}
\end{definition}

We can refer to the following figure \ref{enc-dec} from now on. 
\newline
\begin{figure}[h]
	\centering
	\label{enc-dec}
	\begin{tikzpicture}
         \node[draw] (mes) {$m$};
         \node[draw, right=2cm of mes] (enc) {$E(m)$};
         \node[draw, right=4cm of enc] (chan) {$ch(E(m))$};
         \node[draw, right=2cm of chan] (dec) {$D(ch(E(m))) = m$};
         \draw [->] (mes)--(enc);
         \draw [->] (enc)--(chan);
         \draw [->] (chan)--(dec);
         
         \node (E) at (1.25, 0.25) {$E$};
         \node (channel) at (5.4, 0.25) {$Noisy\text{ }Channel$};
         \node (deco) at (10, 0.25) {$D$};
	\end{tikzpicture}
		\caption{Encoding-Decoding}
\end{figure}


\subsection{Quantifying Error}
\begin{definition}{Hamming Distance}{def:ham-dist}
For any $a, b\in\Sigma^n$, \textit{hamming distance} of $a,b$ is defined as
\[
\Delta(a,b) := \# \{i\in[n] : a_i\neq b_i\}
\]
where $a = (a_1,\ldots, a_n)$ and $b = (b_1, \ldots, b_n)$. We also define \textit{relative hamming distance} as 
\[
\delta (a,b) := \frac{1}{n}\Delta(a,b).
\]
\end{definition}
\begin{remark}
We can verify easily that this $\Delta$ is actually a distance on $\Sigma^n$.
\end{remark}

\begin{definition}{Hamming Weight}{def:hamm-weight}
We define \textit{hamming weight} of an element $v\in\Sigma^n$ as, $wt(c) := \Delta(0,v)$.
\end{definition}

Refering to the above figure \ref{enc-dec}, we define the error for that transmitted codeword to be $\Delta(c,y)$.

\section{Performance of an Error Correcting Code}

\begin{definition}{t-Error Channel \& t-Error Correcting Code}{def:t-error,erasure}
	\begin{itemize}
\item \text{(t-Error Channel)} An $n$ symbol \textit{t-Error Channel} is a relation $ch : \Sigma^n\to \Sigma$ such that $\forall c\in\Sigma^n$, 
\[
\Delta(c, ch(c))\leq t.
\]
\item \text{(t-Error Correcting Code)} Let $\cC\in\Sigma^n$ is a \textit{t-Error Correcting code} if $\forall$ t-Error Channel $ch$ and $\forall m\in[|\cC|]$ ,
\[
D(ch(E(m))) = m.
\]
\end{itemize}

\end{definition}
\begin{example}
$\cC_{3,rep}$ is a 1-error correcting code.
\end{example}
\begin{example}
$\cC_\oplus$ is a 0-error correcting code.
\end{example}

\subsection{Some weaker Notations}
\begin{definition}{t-Error Erasure Channel \& t-Error Detecting Code}{def:err-detect}
       \begin{itemize}
       	\item (t-Erasure Channel) A \textit{t-Erasure Channel} is a mapping $ch : \Sigma^n\to (\Sigma\cup\{?\})^n$, where $?\notin\Sigma$, such that $\forall a\in\Sigma^n$, 
       	\[
       	\Delta(a, ch(a))\leq t.
       	\]
       	and for all $i\in[n]$ such that $a_i\neq ch(a)_i$, we would have $ch(a)_i = ?$.
       	\item (t-Error Detection Code) A code $\cC\subseteq\Sigma^n$ is an \textit{t-Error Detecting code} if there exists a detection procedure $D$ such that $m\in[|C|]$ \& $\forall$ t-Error Channel,
       	\[
       	     D(ch(E(m))) = \mathbbm{1}_{\{ch(E(m)) = E(m)\}}.
       	\]
       	       	       \end{itemize}
\end{definition}

\begin{remark}
      Similarly we can also define a t-Erasure code $\cC\subseteq\Sigma^n$ if $\forall m \in [|C|]$ \& t-Erasure Channel $ch$, 
      \[
      D(ch(E(m))) = E(m) \cong m.
      \]
\end{remark}
\begin{example}

\end{example}

\subsection{Error Correction Capability of $\cC_\oplus$ and $\cC_{3,rep}$}

\section{Distance of a Code}
A parameter for quantifying error correction capability of a code is the distance of that code.
\begin{definition}{Distance of a Code}{def:dist}
For a code $\cC\in\Sigma^n$, we define its \textit{distance} as the following 
\[
d(\cC) := \min_{
	\begin{subarray}
		
		c_1\neq c_2 \\
		c_1, c_2\in\cC
\end{subarray}
} \Delta(c_1, c_2).
\]
\end{definition}
\begin{example}
	$d(\cC_\oplus) = 2$ and $d(\cC_{3,rep}) = 3$.
\end{example}

\begin{proposition}{}{}
	Given a code $\cC$, the followig are equivalent.
	\begin{enumerate}
		\item $\cC$ has minimum distance $d\geq 2$.
		\item If $d$ is odd, then $\cC$ can correct upto $\frac{d-1}{2}$ many errors.
		\item $\cC$ can detect $d-1$ many errors.
		\item $\cC$ can correct $d-1$ many erasures.
	\end{enumerate}
\end{proposition}
\begin{proof}

\end{proof}
